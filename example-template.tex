\documentclass{article}
\usepackage[utf8]{inputenc}
\usepackage{amsmath}
\usepackage{amsfonts}
\usepackage{amssymb}
\usepackage{graphicx}
\usepackage{hyperref}

\title{Collaborative LaTeX Document}
\author{Your Name}
\date{\today}

\begin{document}

\maketitle

\tableofcontents

\section{Introduction}

Welcome to the collaborative LaTeX editor! This is a sample document that demonstrates various LaTeX features.

\section{Mathematical Expressions}

Here are some examples of mathematical expressions:

\subsection{Inline Math}
The quadratic formula is $x = \frac{-b \pm \sqrt{b^2 - 4ac}}{2a}$.

\subsection{Display Math}
The integral of $e^x$ is:
\[
\int e^x \, dx = e^x + C
\]

\subsection{Aligned Equations}
\begin{align}
f(x) &= x^2 + 2x + 1 \\
     &= (x + 1)^2
\end{align}

\section{Lists}

\subsection{Unordered List}
\begin{itemize}
    \item First item
    \item Second item
    \item Third item
\end{itemize}

\subsection{Ordered List}
\begin{enumerate}
    \item First step
    \item Second step
    \item Third step
\end{enumerate}

\section{Tables}

\begin{table}[h]
\centering
\begin{tabular}{|c|c|c|}
\hline
Name & Age & City \\
\hline
Alice & 25 & New York \\
Bob & 30 & London \\
Charlie & 35 & Tokyo \\
\hline
\end{tabular}
\caption{Sample table}
\label{tab:sample}
\end{table}

\section{Code}

Here's an example of inline code: \texttt{console.log("Hello World")}.

For code blocks, you can use the verbatim environment:
\begin{verbatim}
function fibonacci(n) {
    if (n <= 1) return n;
    return fibonacci(n-1) + fibonacci(n-2);
}
\end{verbatim}

\section{References}

You can create references to sections, equations, tables, and figures. For example, see Table~\ref{tab:sample}.

\section{Conclusion}

This template provides a good starting point for your collaborative LaTeX documents. Feel free to modify and expand it according to your needs!

\end{document}